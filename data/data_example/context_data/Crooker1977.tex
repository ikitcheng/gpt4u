\documentclass[10pt]{article}
\usepackage[utf8]{inputenc}
\usepackage[T1]{fontenc}
\usepackage{amsmath}
\usepackage{amsfonts}
\usepackage{amssymb}
\usepackage[version=4]{mhchem}
\usepackage{stmaryrd}
\usepackage{graphicx}
\usepackage[export]{adjustbox}
\graphicspath{ {./images/} }

\title{A Mechanism for Pressure Anisotropy and Mirror Instability in the Dayside Magnetosheath }


\author{N. U. Crooker and G. L. Siscoe}
\date{}


\begin{document}
\maketitle


\section{Department of Atmospheric Sciences, University of California, Los Angeles, California 90024}
\begin{abstract}
The plasma in the dayside magnetosheath exhibits a persistent pressure anisotropy in the sense $p_{\perp}>$ $p_{\parallel}$. A likely source for this anisotropy is the effect of field compression and plasma depletion along field lines as magnetosheath plasma flows toward the magnetopause. The model of Zwan and Wolf describing this effect for the case of isotropic pressure is combined with the double-adiabatic fluid equations to predict the behavior of the anisotropic pressure. For a fluid element following a streamline inward from the bow shock, we find theoretical pressure anisotropies of magnitude large enough to trigger the mirror instability over most of the dayside magnetosheath. These findings are supported by the observations of Kaufmann et al. of large-amplitude hydromagnetic waves in the inner magnetosheath believed to be generated by the mirror instability.
\end{abstract}

\section{INTRODUCTION}
The magnetosheath magnetic field is characterized by fluctuations over a wide range of frequencies (reviewed by Fairfield [1976]), the sources of which are not well understood. In the inner magnetosheath, Kaufmann et al. [1970] report Explorer 12 observations of large-amplitude hydromagnetic waves which fit the criteria of the slow magnetoacoustic mode. They argue that since these waves are not observed in the solar wind and cannot propagate easily outward from the magnetopause, their source must be either at the shock front or within the magnetosheath. Further, they suggest that a likely mechanism for their generation is the mirror instability, which is associated with greater pressure perpendicular to the magnetic field than parallel to it (i.e., $p_{\perp}>p_{\parallel}$ ) [Chandrasekhar et al., 1958]. Such pressure anisotropy has recently been reported from Explorer 33 measurements as a persistent feature of the dayside magnetosheath [Crooker et al., 1976; Crooker, 1976]. The purpose of this brief report is to note that strong pressure anisotropy in the sense $p_{\perp}>p_{\parallel}$ results naturally in the flow in the magnetosheath from an application of the double-adiabatic relations to already existing magnetohydrodynamic (MHD) solutions for the subsonic region.

\section{THE MECHANISM}
As the interplanetary magnetic field (IMF) is convected through the magnetosheath, it is observed to assume a draping orientation around the magnetopause [Fairfield, 1967]. A stagnation streamline calculation by Lees [1964] predicts that the compression of the draped field lines against the dayside magnetopause decreases the plasma density there. Zwan and Wolf [1976] extend the calculation to a large portion of the subsonic region. They show that the reduced density in the stagnation region results from the squeezing of the plasma out of the region along magnetic field lines, a result anticipated by $M$ idgley and Davis [1963]. It is the combination of an increase in field strength and a reduction in density along a flow streamline that gives rise to the $p_{\perp}>p_{\parallel}$anisotropy in double-adiabatic flow.

For steady flow the double-adiabatic relations can be written as

$$
\begin{gathered}
p_{\perp}=c_{1} \rho B \\
p_{\parallel}=c_{2} \rho^{3} / B^{2}
\end{gathered}
$$

Copyright (c) 1977 by the American Geophysical Union.

Paper number 6A0660. where $\rho$ is mass density, $B$ is field intensity, and $c_{1}$ and $c_{2}$ are streamline constants [Chew et al., 1956]. The streamline constants are determined at the bow shock, but the usual shock relations do not give $p_{\perp}$ and $p_{\parallel}$separately. The shock compression could amplify either component of the pressure over the other depending on whether the field is more perpendicular [Noerdlinger, 1964] or more parallel [Kennel and Sagdeev, 1967] to the shock normal. Noerdlinger and Kennel and Sagdeev note that the anisotropy due to shock compression can be unstable to the production of waves. To explore here the differential behavior of the pressure components in the postshock flow, the pressure at the shock is assumed to be isotropic. Parameters there are denoted by a subscript zero. The pressure anisotropy ratio along a streamline in the magnetosheath then varies as

$$
\frac{p_{\perp}}{p_{\parallel}}=\left(\frac{\rho_{0}}{\rho}\right)^{2}\left(\frac{B}{B_{0}}\right)^{3}
$$

Note that the anisotropy is inversely proportional to the density squared and directly proportional to the field magnitude cubed. Plasma is subject to the mirror instability under the condition

$$
\beta_{\perp} / \beta_{\parallel}>1+\left(1 / \beta_{\perp}\right)
$$

where $\beta_{\perp}$ and $\beta_{\parallel}$are the ratios of the gas pressures $p_{\perp}$ and $p_{\parallel}$to the magnetic pressure $B^{2} / 8 \pi$

In order to obtain a quantitative estimate of the anisotropy resulting from field compression and plasma depletion in the dayside magnetosheath, we substitute the field and density variations along the stagnation streamline from the $M H D$ model of $Z$ wan and Wolf [1976] into (3). The more rigorous approach, which we shall not attempt, would be to incorporate (1) and (2) directly into the model of Zwan and Wolf in place of their isotropic pressure assumption. However, it is clear that if a self-consistent calculation showed that the pressure anisotropy is small, then the isotropic pressure assumption of Zwan and Wolf would be a good approximation to the dynamics, and we should recover the same result by our procedure. Conversely, if we find a large pressure anisotropy, then the self-consistent calculation must also give a large anisotropy. We therefore expect our procedure to give a rough estimate of the self-consistent value.

The model variations of $B / B_{0}$ and $\rho / \rho_{0}$ used in the substitution are shown as dashed lines in Figure 1. They were derived by Zwan and $W$ olf for a solar wind field perpendicular to the earth-sun line and the earth's dipole, an Alfvén Mach

\begin{center}
\includegraphics[max width=\textwidth]{2023_03_19_bf21a1c81fec79e6b721g-2}
\end{center}

Fig. 1. Pressure anisotropy $\beta_{\perp} / \beta_{\parallel}$in the magnetosheath along the stagnation streamline calculated from the double-adiabatic formalism and the model density and magnetic field variation $\rho / \rho_{0}$ and $B / B_{0}$ (dashed lines) of $Z$ wan and Wolf [1976]. The subscript zero refers to values just inside the bow shock. The solid lines show that the mirror instability criteria $\beta_{\perp} / \beta_{\parallel}>1+\left(1 / \beta_{\perp}\right)$ is satisfied along most of the stagnation streamline.

number $M_{A}=8$, and a specific heat ratio $\gamma=f$. The solid lines are the resulting variations of the terms in the mirror instability inequality (4). The figure shows that the criteria for instability along the stagnation streamline are satisfied throughout $95 \%$ of the magnetosheath.

\section{Discussion}
The results in Figure 1 suggest that field compression and plasma depletion are responsible for the persistent pressure anisotropy and the large-amplitude hydromagnetic waves in the dayside magnetosheath. Although the estimate of anisotropy is for the stagnation streamline, where field compression and plasma depletion are maximum, $Z$ wan and Wolf [1976] show that these two effects are large across the entire dayside magnetosheath.

Factors which tend to decrease the amount of anisotropy indicated in Figure 1 are (1) other field orientations, (2) higher solar wind Alfvén Mach numbers, (3) magnetic merging, and (4) dissipation by wave growth. With regard to the first factor, although IMF oriented parallel to solar wind flow expands rather than compresses in traversing the magnetosheath, the expansion is confined to a small region around the stagnation point [Spreiter et al., 1966]. For the average spiral orientation [Ness et al., 1971], Spreiter et al. [1966] show that compression occurs across the dayside magnetosheath. With regard to the third factor, magnetic merging at the magnetopause tends to decrease the plasma depletion, as is discussed by $Z$ wan and Wolf [1976]. However, the pressure anisotropy might have an interesting effect on the merging rate, as has been discussed by Hill [1975]. For the usual sense of anisotropy $\left(p_{1}>p_{\parallel}\right)$, the merging rate is predicted to increase. With regard to the fourth factor, although a quantitative evaluation of the effect of dissipation by wave growth lies outside the scope of this paper, a major conclusion that can be drawn from Figure 1 is that double-adiabatic flow forces rapid growth of the $p_{\perp}>p_{\parallel}$ anisotropy throughout the subsonic magnetosheath, providing over the same region a means for transferring flow energy into wave energy through the mechanism of the mirror instability. This conclusion is supported by the previously cited observations of waves in the dayside magnetosheath. The action of the waves to reduce the pressure anisotropy might be the process which limits the rate of energy transfer. We note that the first three factors mentioned could reduce the anisotropy in Figure 1 by a large factor and still enough anisotropy would remain to trigger the mirror instability.

Pressure anisotropy in the observed sense $p_{1}>p_{\parallel \mid}$can also trigger other plasma instabilities, which may be responsible for some of the fluctuations observed in the magnetosheath. For example, it can trigger the well-known cyclotron instability [Lin and Parks, 1974, and references therein], which may be responsible for magnetosheath lion roars [Smith and $T_{\text {suru- }}$ tani, 1976]. It can also trigger the electrostatic instability, which produces waves at $\left(n+\frac{1}{2}\right)$ multiples of the gyrofrequency, where $n=0,1,2, \cdots$, and which has been associated with electron-generated VLF emissions in the magnetosphere [Kennel et al., 1970; Coroniti et al., 1971; Young, 1974], and thus produce another source of magnetosheath waves.

\end{document}